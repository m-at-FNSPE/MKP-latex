\documentclass[../main.tex]{subfiles}
\graphicspath{{\subfix{../images/}}}
\begin{document}

\begin{theorem}[Minkowského nerovnost]
    \begin{equation}\label{eq:Mink}
        \left ( \int_\Omega |f(x) + g(x)|^p dx \right ) ^{\frac{1}{p}} \leq \left ( \int_\Omega |f(x)|^p dx \right ) ^{\frac{1}{p}} + \left ( \int_\Omega |g(x)|^p dx \right ) ^{\frac{1}{p}}
    \end{equation}
\end{theorem}


\begin{theorem}[Holderova nerovnost]
    Nechť $\left (  \frac{1}{p} + \frac{1}{q} = 1 \right ), f \in \mathbb{L}_p(\Omega), g \in \mathbb{L}_q(\Omega)$
    \begin{equation}\label{eq:Holder}
        \int_\Omega \left| f(x)g(x) \right| dx \leq \left ( \int_\Omega |f(x)|^p dx \right ) ^{\frac{1}{p}} * \left ( \int_\Omega |g(x)|^q dx \right ) ^{\frac{1}{q}}
    \end{equation}
\end{theorem}



\begin{theorem}[Vnoření $L_p$ prostorů]
    


    $p_2 > p_1 \geq 1 \implies \mathbb{L}_{p_2}(\Omega) \subset \mathbb{L}_{p_1}(\Omega)$
    \end{theorem}
    



    \begin{theorem}[Věta o stopách]
        $\Omega$ - omezená oblast, $\partial \Omega$ - Liepschitzovská
        
        $\exists_1 T: \mathbb{W}_2^{(1)}(\Omega) \mapsto \mathbb{L}_p (\partial \Omega), \text{omezený lineární operátor tak, že}$
        
        $ \forall f\in C (\bar \Omega), Tf = f /_{\partial \Omega} $
    \end{theorem}
    



    \begin{theorem}[Lax-Milgramova věta]
        Nechť V je Hilbertův prostor, $a(\cdot, \cdot)$ je bilineární forma, F je spojitý lineární funkcionál na V a platí:
        \begin{enumerate}
            \item $a(\cdot, \cdot)$ je omezená: $(\exists K>0)(\forall u,v \in V)(|a(u,v)|\leq K ||u|| ||v||)$
            \item $a(\cdot, \cdot)$ je V-eliptická: $(\exists\alpha>0)(\forall v \in V)(a(u,u)\geq \alpha||u||^2)$
        \end{enumerate}
        Pak $(\exists_1z\in V)(\forall v \in V)(a(z,v) = F(v))$
    \end{theorem}



    \begin{theorem}[Céova]
    

        Nechť pro $a(.,.)$ a $F$ platí výše uvedené předpoklady a $z$ řeší slabou formulaci úlohy . Pak pro řešení $z_h$ úlohy Galerkinovou metodou platí 
        \begin{equation}
            ||z_h - z||_V \leq \frac{K}{\alpha} min\{||z - v||_V | v\in V_h \}
        \end{equation} 
        \end{theorem}





        \begin{theorem}[MKP1]
            Množina $\bar{\Omega}$ (z okrajové úlohy) je rozdělena triangulací $\mathcal{T}_h$ na konečný počet podmmnožin K (oblastí konečných prvků) tak, že 
            \begin{enumerate}
                \item $\bar{\Omega} = \bigcup_{K\in\mathcal{T}_h} \bar{K}$
                \item $(\forall K  \in \mathcal{T}_h)(K\neq\emptyset \text{ a K je oblast})$
                \item $(\forall K_1, K_2 \in \mathcal{T}_h)(K_1 \neq K_2 \implies K_1 \cap K_2 = \emptyset)$
                \item $(\forall K \in \mathcal{T}_h) (\partial K \text{ je Lipschitzovská} )$
            \end{enumerate}
        \end{theorem}
        
        \begin{theorem}[MKP2]
            Na každé množině $K\in\mathcal{T}_h$ definujeme vhodné funkce sloužící k aproximaci řešení variační úlohy (slabé formulace) . Tyto funkce jsou polynomy, nebo "blízké" polynomům
        \end{theorem}
        
        
        \begin{theorem}[MKP3]
            Aproximaci řešení variační úlohy (slabé formulace)  hledáme pomocí bazických funkcí, jejichž nosič je co nejmenší při zachování shodného popisu tvaru těchto funckí. 
        \end{theorem}
        

        \begin{lemma}[O redukci]
            Nechť $\mathcal{P}$ je polynom stupně $\mathcal{D} \geq 1$ v $n$ proměnných, který je roven $0$ v nadrovině $V$, která je popsána funkcionálem $L\in(\mathbb{R}^n)^\#$ (tj. $V\equiv \left\{x\in\mathbb{R}^n| L(x)= 0\right\}$).
            Pak existuje polynom $Q$ stupně $\mathcal{D} - 1$ tak, že $(\forall x\in\mathbb{R}^n)(P(x) = L(x)*Q(x))$    
        \end{lemma}




































\end{document}