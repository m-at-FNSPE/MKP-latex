\documentclass[../main.tex]{subfiles}
\graphicspath{{\subfix{../images/}}}
\begin{document}


\subsection{Úkol 1  - Příklad 9}
Určete prvky matice lineární soustavy rovnic pro okrajovou úlohu řešenou\\ metodou konečných prvků
\begin{equation}
    -(\lambda(x)u')' + q(x)u = f(x) \text{ v } (a,b)
\end{equation}
\begin{equation}
    u|_{x=a} = 0, u|_{x=b} = 0
\end{equation}
kde $a=2, b=3, \lambda(x) = 1+x, q(x)=2$ a síť MKP je rovnoměrná a daná uzly:
\begin{equation}
    x_j = a + jh, h=\frac{b-a}{m}
\end{equation}
kde $m\in \mathbb{N}$ je počet prvků diskretizující interval $(a,b)$, který je obecným parametrem.


\subsubsection{Vzorce použité z přednášky}
\begin{multline}\label{pr1:middle}
    a(\phi_j, \phi_j) = \int_0^1 p((x_j - x_{j-1})y + x_{j-1}) \frac{dy}{x_j - x_{j-1}} +\\+ \int_0^1 q((x_j - x_{j-1})y + x_{j-1}) y^2 (x_j - x_{j-1}) \ dy -\\- \int_0^1 p(-(x_{j+1} - x_j)y + x_{j+1}) \frac{dy}{ x_{j+1} - x_j} -\\- \int_0^1 q(-(x_{j+1} - x_j)y + x_{j+1}) y^2 (x_{j+1} - x_j) \ dy
\end{multline}
\begin{multline}\label{pr1:bellow}
    a(\phi_j, \phi_{j-1}) = - \int_0^1 p((x_j - x_{j-1})y + x_{j-1}) \frac{dy}{x_j - x_{j-1}} -\\- \int_0^1 q((x_j - x_{j-1})y + x_{j-1}) y(1-y) (x_j - x_{j-1}) \ dy
\end{multline}
\begin{multline}\label{pr1:above}
    a(\phi_j, \phi_{j+1}) = - \int_0^1 p(-(x_{j+1} - x_j)y + x_{j+1}) \frac{dy}{ x_{j+1} - x_j} -\\- \int_0^1 q(-(x_{j+1} - x_j)y + x_{j+1}) y(1-y) (x_{j+1} - x_j) \ dy
\end{multline}


\subsubsection{Řešení}
Ze zadání vidíme následující užitečné vztahy:
\begin{equation}\label{pr1:vztahhm}
    h=\frac{1}{m}
\end{equation}
\begin{equation}\label{pr1:h}
    x_j - x_{j-1} = h = x_{j+1} - x_j
\end{equation}
\begin{equation}\label{pr1:-2h}
    x_{j-1}- x_{j+1} = -2h
\end{equation}


Dosazením zadání do \eqref{pr1:middle} a použitím \eqref{pr1:h} dostaneme:
\begin{multline}
    a(\phi_j, \phi_j) = \int_0^1 \left( 1 + hy + x_{j-1}\right) \frac{dy}{h} + \int_0^1 2 y^2 (h) \ dy -\\- \int_0^1 (1 -hy + x_{j+1}) \frac{dy}{ h} - \int_0^1 2 y^2 (h) \ dy
\end{multline}
\begin{equation}
    a(\phi_j, \phi_j) = \int_0^1 \frac{1}{h}\left( 1 + hy + x_{j-1}\right)  +  2h y^2   -  \frac{1}{ h}(1 -hy + x_{j+1})  - 2h y^2 \ dy
\end{equation}
\begin{equation}
    a(\phi_j, \phi_j) = \int_0^1 \frac{1 + x_{j-1}}{h}  + y +  2h y^2   -  \frac{1+ x_{j+1}}{ h} + y   - 2h y^2 \ dy
\end{equation}
\begin{equation}
    a(\phi_j, \phi_j) = \int_0^1 \frac{x_{j-1}- x_{j+1}}{h}  + 2y +  0 y^2 \ dy
\end{equation}
Použijeme vztah \eqref{pr1:-2h}
\begin{equation}
    a(\phi_j, \phi_j) = \int_0^1 \frac{-2h}{h}  + 2y \ dy
\end{equation}
\begin{equation}
    a(\phi_j, \phi_j) = \int_0^1 -2  + 2y \ dy
\end{equation}
\begin{equation}
    a(\phi_j, \phi_j) = -2  + 1 = -1
\end{equation}
Stejným způsobem spočtěme \eqref{pr1:bellow}:
\begin{equation}
    a(\phi_j, \phi_{j-1}) = - \int_0^1 (1+hy + x_{j-1}) \frac{dy}{h} - \int_0^1 2 y(1-y) (h) \ dy
\end{equation}

\begin{equation}
    a(\phi_j, \phi_{j-1}) = - \int_0^1 \frac{1+ x_{j-1}}{h} + y  + 2h y(1-y) \ dy
\end{equation}

\begin{equation}
    a(\phi_j, \phi_{j-1}) = - \int_0^1 \frac{1+ x_{j-1}}{h} + (1+2h)y - 2h y^2 \ dy
\end{equation}

\begin{equation}
    a(\phi_j, \phi_{j-1}) = -  \frac{1+ x_{j-1}}{h} - \frac{(1+2h)}{2} + \frac{2h}{3}
\end{equation}
Dosaďme za $x_{j-1} = 2 + (j-1)h$ : 
\begin{equation}
    a(\phi_j, \phi_{j-1}) = -  \frac{3}{h} - (j-1) - \frac{(1+2h)}{2} + \frac{2h}{3}
\end{equation}
\begin{equation}
    a(\phi_j, \phi_{j-1}) = -j + \frac{1}{2} -\frac{3}{h} -\frac{h}{3}
\end{equation}
Nakonec ještě použijeme \eqref{pr1:vztahhm}:
\begin{equation}
    a(\phi_j, \phi_{j-1}) = -j + \frac{1}{2} -3m -\frac{1}{3m}
\end{equation}


Pro \eqref{pr1:above} buď využijme toho že matice je symetrická nebo jednoduše spočtěme:
\begin{equation}
    a(\phi_j, \phi_{j+1}) = - \int_0^1 (1 -hy + x_{j+1}) \frac{dy}{h} - \int_0^1 2h y(1-y) \ dy
\end{equation}
\begin{equation}
    a(\phi_j, \phi_{j+1}) = - \int_0^1 \frac{(1 -hy + x_{j+1}) }{h} + 2h y(1-y) \ dy
\end{equation}
\begin{equation}
    a(\phi_j, \phi_{j+1}) = - \int_0^1 \frac{(1 + x_{j+1}) }{h} +(2h - 1)y - 2hy^2 \ dy
\end{equation}
Opět dosaďme za $x_{j+1} = 2 + (j+1)h$ : 

\begin{equation}
    a(\phi_j, \phi_{j+1}) = - \int_0^1 \frac{(3 + (j+1)h) }{h} + (2h - 1)y - 2hy^2 \ dy
\end{equation}
\begin{equation}
    a(\phi_j, \phi_{j+1}) = - \left( \frac{(3 + (j+1)h) }{h} + \frac{2h - 1}{2} - \frac{2h}{3}\right)
\end{equation}
\begin{equation}
    a(\phi_j, \phi_{j+1}) = - \frac{(3 + (j+1)h) }{h} - \frac{1}{2} + \frac{2h}{6} 
\end{equation}

\begin{equation}
    a(\phi_j, \phi_{j+1}) = - \frac{3}{h} - j - \frac{1}{2} + \frac{2h}{6} 
\end{equation}


A opět použijeme \eqref{pr1:vztahhm}:
\begin{equation}
    a(\phi_j, \phi_{j+1}) = - 3m - j - \frac{1}{2} + \frac{1}{3m} 
\end{equation}

\subsubsection{Výsledek}
Nakonec tedy dostáváme:

\begin{equation}
    a(\phi_j, \phi_j) = -1
\end{equation}
\begin{equation}
    a(\phi_j, \phi_{j-1}) = -j + \frac{1}{2} -3m -\frac{1}{3m}
\end{equation}
\begin{equation}
    a(\phi_j, \phi_{j+1}) = - 3m - j - \frac{1}{2} + \frac{1}{3m} 
\end{equation}


\end{document}