\documentclass[../main.tex]{subfiles}
\graphicspath{{\subfix{../images/}}}
\begin{document}

\todo{Chybi cele lol}

\begin{proof}
    \todo{chybi prvni cast}

    Pro druhou část tvrzení:

    $\mathbb{W}_\infty^{(r+1)}(\Omega) = \left\{v\in _\infty(\Omega)| (\forall |\alpha|\leq r+1)(\mathcal{D}^\alpha v \in L_\infty)\right\}$,

    $L_\infty(\Omega) = \left\{v: (\Omega)\mapsto \mathbb{R}| (\exists K > 0)(S\forall x \in \Omega)(|v(x)| \leq K) \right\} $

    Provedeme pro $r = 0$ (Lagrange), víme: $\mathcal{Y}_\mathcal{J} u \in \mathcal{C}^{(0)}$ z předchozí vlastnosti a $\mathcal{Y}_\mathcal{J} u (x) = \mathcal{Y}_K u(x)$ pro $x\in\bar{K}, K \in \mathcal{J}$  a $\mathcal{Y}_K u(x)\in \mathcal{P}_K$ (tj je polynom), tj. $\mathcal{Y}_K u(x)\in C^{(\infty)}(\bar{K})$

    Prozkoumáme možnost derivování $\mathcal{Y}_\mathcal{J} u$ v $\Omega$ (pomocí derivací ve smyslu  distribucí)

    Pro $k\in\hat{n}:$ 
    \begin{multline}
    <\partial_{x_k}>(\mathcal{Y}_\mathcal{J} u), \phi> = - <\mathcal{Y}_\mathcal{J} u, \partial_{x_k} \phi> = - \int_\Omega \mathcal{Y}_\mathcal{J} u(x) \partial_{x_k}\phi(x) \ dx =\\
    %= - \sum_{K\in\mathcal{J}}\int_{K} \mathcal{Y}_K u(x) \partial_{x_k} \phi(x) \ dx =  \left{\text{Green}\right} = \sum_{K\in\mathcal{J}} \left\{- \int_{\partial K} \mathcal{Y}_K u(x) \phi(x) N^{(K)}_k (x) \ dS + \int_{K} \partial_{x_k} \mathcal{Y}_K u(x) \phi(x) \ dx\right\} =\\
    %= \left\{\text{Platí že} \vec{N}^{K_1} = - \vec{N}^{K_2} \text{viz obrázek}\right\} = \left\{ \mathcal{Y}_{K_1} u(x)|_{\partial K_1 \cap \partial K_2} = \mathcal{Y}_{K_2} u(x)|_{\partial K_1 \cap \partial K_2}, \phi|_{\partial\Omega} = 0  \right\}  = \left\{\text{Tj první člen je nula}\right\} = \\
    = \sum_{K\in\mathcal{J} }\int_K \Chi_{k} (x) \partial_{x_k} \mathcal{Y}_K u(x) \phi(x) \ dx = \int_\Omega\sum_{K\in\mathcal{J} } \Chi_{k} (x) \partial_{x_k} \mathcal{Y}_K u(x) \phi(x) \ dx = \left\{\text{Pro K omezená jsme v }L_\infty\right\} = \left\{ původně bylo \right\} =\\
    = <\partial_{x_k}\mathcal{Y}_\mathcal{J} u, \phi >
    \end{multline}


    $\implies \partial_{x_k} \mathcal{Y}_\mathcal{J} u \in L_\infty(\Omega), tj. \mathcal{Y}_\mathcal{J} u \in \mathbb{W}_\infty^{(1)}(\Omega)$

\end{proof}


\begin{remark}
    Pro $\mathcal{J}$ triangulaci $\Omega $:
    \todo{obrázek}
\end{remark}


\end{document}